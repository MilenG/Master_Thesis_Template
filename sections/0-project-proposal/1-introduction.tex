\section{Introduction}
\label{sec:introduction}
% Mention scientific field, problem statement, and the research gap you wish to address. 
This document is \textbf{not} compulsory and presents a way to present a proposal for a thesis outside the Marketplace. This \textbf{Proposal} allows you to present what you would like to work on. It should be immediately clear how your research is scientifically relevant. Make clear to which key papers you will compare your eventual results. 
\TODO{This is a TODO} This is a test citation \cite{Gruber1995} 

Towards the end of the introduction, you should also add your \textit{preliminary} \textbf{reasearch questions (RQ)} here. You may want to state your main RQ like this:

\noindent\textit{To what extent can a master thesis template enhance the quality of the final thesis?}\REMARK{This is a remark}
You can then list the sub-questions as:
\begin{itemize}
    \item How does the structure of the template influence the final grading?
    \item To what extent is textual guidance sufficient for structured working?
    \item \dots
\end{itemize}